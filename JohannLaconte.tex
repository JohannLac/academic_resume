\documentclass[11pt,a4paper,sans]{moderncv}        % possible options include font size ('10pt', '11pt' and '12pt'), paper size ('a4paper', 'letterpaper', 'a5paper', 'legalpaper', 'executivepaper' and 'landscape') and font family ('sans' and 'roman')
\usepackage[
backend=biber, 
bibencoding=utf8,
style=ieee, 
sorting=ynt, 
natbib=true, 
doi=false, 
isbn=false, 
url=false, 
eprint=false, 
maxcitenames=1, 
mincitenames=1,
]{biblatex}

\bibliography{biblio}

% moderncv themes
\moderncvstyle{classic}                             % style options are 'casual' (default), 'classic', 'banking', 'oldstyle' and 'fancy'
\moderncvcolor{blue}                               % color options 'black', 'blue' (default), 'burgundy', 'green', 'grey', 'orange', 'purple' and 'red'

% adjust the page margins
\usepackage[scale=0.85]{geometry}
%\setlength{\hintscolumnwidth}{3cm}                % if you want to change the width of the column with the dates
%reduce spacing after quote
 \patchcmd{\makecvhead}%
 {\hfill\null\\[2.5em]}%
 {\hfill\null\\[1em]}%
 {}%
 {}%

% personal data
\name{Johann}{Laconte}
\title{Ph.D. in Robotics}                               % optional, remove / comment the line if not wanted
\address{1 rue Bonnabaud}{63000 Clermont-Ferrand}{France}% optional, remove / comment the line if not wanted; the "postcode city" and "country" arguments can be omitted or provided empty
\phone[mobile]{+33 (0)7 82 21 02 20}                   % optional, remove / comment the line if not wanted; the optional "type" of the phone can be "mobile" (default), "fixed" or "fax"
\email{laconte.johann@gmail.com}                               % optional, remove / comment the line if not wanted

\extrainfo{Updated on \today}                 % optional, remove / comment the line if not wanted

%   to show numerical labels in the bibliography (default is to show no labels)
%\makeatletter\renewcommand*{\bibliographyitemlabel}{\@biblabel{\arabic{enumiv}}}\makeatother
\renewcommand*{\bibliographyitemlabel}{[\arabic{enumiv}]}

%\quote{\textbf{Research interests:} Robotics; Applied Mathematics; Artificial Perception; Risk Assessment; Intelligent Vehicles; Mapping}
%----------------------------------------------------------------------------------
%            content
%----------------------------------------------------------------------------------
\begin{document}
%-----       resume       ---------------------------------------------------------
\makecvtitle
\vskip-2em
\textbf{Research Interests:} Robotics; Applied Mathematics; Traversability; Safety Analysis; State Estimation; Mapping.
\section{Employment}
\cventry{10 2023-- now}{Junior Research Chair (Tenure-Track)}{INRAE - National Research Institute for Agriculture, Food and Environment, France}{}{}{Perception and navigation in unstructured, deformable environments}  % arguments 3 to 6 can be left empty
\section{Education}
\cventry{09 2022-- 09 2023}{Postdoc in Robotics}{University of Toronto, Canada}{}{}{Development of safety analysis techniques for the certification of localization algorithms. Supervision of several research projects around state estimation.\\ \textbf{Supervisor:} Tim Barfoot}  % arguments 3 to 6 can be left empty

  \cventry{03 2022--\\09 2022\hskip.4em}{Postdoc in Robotics}{Université Laval, Canada}{}{}{Supervision of several research projects around field robotics in nordic environments. \\ \textbf{Supervisor:} François Pomerleau}  % arguments 3 to 6 can be left empty

  \cventry{2018--2021}{Ph.D. in Robotics}{Clermont Auvergne University (UCA), France; Université Laval, Canada}{}{}{Development of a theoretical framework for meaningful risk assessment in occupancy grids. \\ \textbf{Supervisors:} Romuald Aufrère (UCA), François Pomerleau (Université Laval), Roland Chapuis (UCA), Christophe Debain (National Research Institute for Agriculture, Food and the Environment)}  % arguments 3 to 6 can be left empty

  \cventry{2017--2018}{Master Degree in Robotics}{Clermont Auvergne University}{}{}{Ranked 1/24.}

  \cventry{2015--2018}{Engineering Degree in Computer Science and Modeling}{Institut Supérieur d'Informatique, de Modélisation et de leurs Applications}{}{}{Ranked 2/120.}

\section{Editorial activities}
\cventry{2023}{Associate Editor}{}{}{}{
  IEEE/RSJ International Conference on Intelligent Robots and Systems (IROS)
}
\cventry{2018--now}{Reviewing Services}{}{}{}{
	Recurrent reviewer for ICRA, IROS and RA-L.
}
\section{Professional activities}
\cventry{2021}{Research Internship}{Université Laval}{Quebec City, Canada}{\textit{2 Months}}{
	Collaboration with the Northern Robotics Laboratory (Norlab), leading to the publications of \citet{baril2021kilometer} and \citet{Laconte2021IVloc}.
}
\cventry{2021}{I-SITE IMOBS3 Research Grant representative}{}{}{}{
	Ph.D. student representative of the I-SITE Clermont label, granting 10M euros per year for the research institute.
}
\cventry{2018--2021}{Organization of seminars}{}{}{}{
	Organization of various seminars in the research department.
}
\cventry{2020}{Research Internship}{Université Laval}{Quebec City, Canada}{\textit{2 Months}}{
	Collaboration with the Northern Robotics Laboratory (Norlab), leading to the publications of \citet{baril2020evaluation} and \citet{vaidis2020improving}.
}
\cventry{2019}{Winter School}{National Institute for Research in Digital Science and Technology (INRIA)}{Sophia Antipolis, France}{\textit{1 Week}}{
	Winter school covering the basics in both mobile and manipulative robotics.
}
\cventry{2018}{Research Internship}{Université Laval}{Quebec City, Canada}{\textit{5 Months}}{
	Investigation of the measurements bias coming from a light detection and ranging (lidar) sensor.
	Modeling of the return waveform and design of an experimental setup.
	Lead to the publication of \citet{laconte2019lidar}.
}
\cventry{2017}{Internship}{Thales}{Elancourt, France}{\textit{5 Months}}{
Evaluations and improvements of state-of-the-art LIDAR Simultaneous Localization And Mapping (SLAM) algorithms.
}
\cventry{2016--2018}{Robotics Competitions}{}{}{}{
	I took part in several national and international robotics competitions (Robot Challenge, French Robot Cup, \textit{La Nuit du Hack}, \textit{Reconnaissance des Formes et Intelligence Artificielle}).
}

\section{Grants and Distinctions}
\cventry{2022}{Best Ph.D. Thesis Award (2$^{\textbf{nd}}$ place)}{GDR Robotique}{}{}{French national competition of the best Ph.D. thesis in the field of robotics.}
\cventry{2021}{Relève étoile Louis-Berlinguet Award}{FRQNT, Canada}{}{}{For the paper: \citetitle{baril2021kilometer}~\cite{baril2021kilometer}}
\cventry{2020}{Best Robot Vision Paper Award}{Conference on Robots and Vision (CRV)}{}{}{For the paper: \citetitle{baril2020evaluation}~\cite{baril2020evaluation}}
\cventry{2020}{Finalist for Best Student Paper Award}{International Conference on Control, Automation, Robotics and Vision (ICARCV)}{}{}{For the paper: \citetitle{kasmi2020information}~\cite{kasmi2020information}}
\cventry{2018}{Doctoral Research Grant}{\textit{Innovative Mobility: Smart and Sustainable Solutions (IMOBS3) Program}}{}{}{}
\cventry{2018}{Graduate Research Grant}{\textit{WOW! Wide Open to the World Program from I-Site CAP2025 project}}{}{}{}

\section{Research Funding}
\cventry{2023}{Canada - NOVA, FRQNT-NSERC PROGRAM for early-career researchers}{\textit{HUNTER: Highlight the Unexpected: Navigation Through Extreme Regions}. Joint deployments in subarctic regions with Université Laval and the University of Toronto}{approx. 200k€}{}{}


\section{Languages}
\cvdoubleitem{\textbf{English}}{Fluent, TOEIC certificate}{\textbf{French}}{Native Speaker}
\cvdoubleitem{\textbf{Chinese}}{Basic Level, HSK2 certificate}{\textbf{German}}{Notions}

\section{Teaching}
\cventry{2018--2021}{Digital Signal Processing}{\textit{Graduate course}}{}{}{Graduate course about Discrete Fourier Transform, Z transform, signal filtering and their applications.}
\cventry{2018--2021}{Control Theory}{\textit{Graduate course}}{}{}{Graduate course about Laplace transform, regulation, modeling and analysis of continuous systems.}
\cventry{2018--2021}{Projects Supervision}{\textit{Graduate students}}{}{}{Supervision of four robotics graduate projects of 60 or 120 hours per person.}
\cventry{2020-now}{Mentoring}{Ph.D. student}{}{}{Mentoring of several graduate students in various fields of robotics}

\newpage
\renewcommand{\refname}{Scientific Publications}
\nocite{*}
%\AtNextBibliography{\small}
\printbibliography
\end{document}
